% LaTeX file for resume
% This file uses the resume document class (res.cls)

\documentclass[margin]{res}
% the margin option causes section titles to appear to the left of body text
\textwidth=5.2in % increase textwidth to get smaller right margin
%\usepackage{helvetica} % uses helvetica postscript font (download helvetica.sty)
%\usepackage{newcent}   % uses new century schoolbook postscript font
\usepackage{hyperref}
\usepackage{setspace}
\usepackage{graphicx}
\begin{document}

% Center the name over the entire width of resume:
 \moveleft.5\hoffset\centerline{\large\bf Ruichang Zhang}
 
\begin{resume}
\setlength{\parskip}{0.1em}
\section{Contact}
%\begin{spacing}{1}
\textbf{University:}~{\bf Fudan University}, Shanghai\hfill \textbf{Job Intension}:~Data Mining Engineer\\
\textbf{Email}:~zhangruichang112@gmail.com \hfill  \textbf{Phone}:~(+86) 150-0022-6997\\
\textbf{Blog}: \url{http://richardzhang.info} \hfill \textbf{Graduation Date}:~Jun. 2015\\
\textbf{Personal Page}: \url{http://admis.fudan.edu.cn/~rczhang}
%\end{spacing}

\section{work experience}
iQiyi Inc, Shanghai \hfill  Jun. 2015$\sim$ 现在
\begin{itemize} \itemsep -1pt %reduce space between items
\item Big data, recommendation group, data mining engineer
\item CF, LR, GBDT based on hadoop for user personalized recommendation
\item python, java for calculating, shell for pipeline control, hadoop/hive/hbase
\end{itemize}

\section{Education}
 {\bf Fudan University}, Master Degree in Computer Science\hfill Sep.2012$\sim$Jun.2015
 \begin{itemize} \itemsep -1pt  % reduce space between items
 \item Shanghai Key Lab of Intelligent Information Processing, \textbf{TA of C language programming}, Advisor: Prof. Zhou Shuigeng
 \end{itemize}
 {\bf Shanghai University}, Bachelor Degree in Computer Science\hfill Sep.2008$\sim$ Jun.2012
\begin{itemize} \itemsep -1pt %reduce space between items
\item GPA:~\textbf{3.678}/4, Rank:~\textbf{1}/227
%\item GPA Ranking: 1/227
\end{itemize}

\section{Research and Interests}
data mining clustering algorithm, recommendation, and distributed machine learning



\section{Internship and Experience}
%\begin{spacing}{1.2}
\textbf{Nanyang Technological University}, Singapore \hfill Mar.2014$\sim$Aug.2014
%\begin{spacing}{0.03}
%\end{spacing}
\begin{itemize}
\item \textbf{Data mining project}. Extract features related to synthetic lethality pairs(SLs). As synthetic lethality in human cancers are limited, and reliable negative samples are not available, exploit \textbf{PU learning} to train positive samples and test on abundant SLs in yeast to validate the effectiveness of features and PU learning method.
\end{itemize}
\begin{spacing}{1.5}
\end{spacing}
\textbf{iQiyi Data Mining Engineer}, Shanghai \hfill Mar.2015 $\sim$ Jun. 2015
\begin{itemize}
\item shell, python for trasferring data from MySQL to Hive,cronrab as timer
\item generate movie topics
\end{itemize}

\section{Projects}
Metagenomic Sequences Binning \hfill Sep.2013 $\sim$Mar.2014
\begin{spacing}{0.02}
\end{spacing}
\begin{itemize}
\item Exploit \textbf{SKWIC algorithm} to cluster metagenomic reads. SKWIC is a variant of K-means, which takes the weight of each feature to each cluster into consideration, thus boosting the performance of clustering. Paper is published on IEEE journal \emph{TCBB}.
\item \url{https://github.com/zhangruichang/MCluster}
\item Exploit \textbf{topic model-LDA}. Firstly, extracting $k$-mer frequencies from sequences, then transforming reads from $k$-mer space to topic space, and finally clustering reads with SKWIC algorithm. Paper was accepted by a conference and \emph{BMC Bioinformatics}.
\item Exploit \textbf{Deep learning}. Learn features from $k$-mer space based on AutoEncoder, and finally cluster reads with SKWIC algorithm.
\end{itemize}
User Behavior Collecting and Analysing System \hfill Feb.2012$\sim$Jun.2012
\begin{spacing}{0.1}
\end{spacing}
\begin{itemize}
\item Using C$\sharp$, monitor user's actions on IE browser, analyze user's real browsing time on a page, and extract the content, including a \textbf{web page content extraction algorithm} inside. The algorithm first removes html tags which are not related to content, and defines the concept of line block size as the number of characters in several continuous lines, and extract the densest area as approximately content is the densest area in the html file.
\item \url{https://github.com/zhangruichang/UserBehaviorCollect}
\end{itemize}


\section{Honors}
%\begin{spacing}{1}
\textbf{VMware Award}, \emph{Fudan University} \hfill2014\\
\textbf{1st Prize in EMC Guru Contest}, \emph{EMC Corp.} \hfill2014\\
\textbf{Tencent Innovation Award}, \emph{Fudan University} \hfill2013\\
\textbf{First Class Scholarship}, \emph{Fudan University}\hfill2013\\
\textbf{Shanghai Outstanding Graduate Student}, \emph{Shanghai University}\hfill 2011\\
\textbf{Excellent League Member,Excellent Student}, \emph{Shanghai University}\hfill 2011\\
\textbf{Guanghua Scholarship}, \emph{Shanghai University}\hfill 2011\\
\textbf{Special Scholarship}, \emph{Shanghai University}\hfill 2011\\
\textbf{Second Prize}, \emph{National Mathematics Contest for Undergraduate Student}\hfill 2009\\



\section{Skills}
%\begin{spacing}{1}
\textbf{Computer Languages}:~C++ $>$ C $>$  Python $>$ Shell = Matlab = Java = C$\sharp$, etc\\
\textbf{Tools $\&\&$ Environments}:~Linux, \LaTeX, Git, SVN, Hadoop, Hive, etc\\
\textbf{Foreign Language}:~English, CET4 578, CET6 554\\


%\end{spacing}



%\end{spacing}

%\end{spacing}
% Tabulate Computer Skills; p{3in} defines paragraph 3 inches wide

\end{resume}
\end{document}


